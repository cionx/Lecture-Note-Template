\chapter{Feature Tests}

\nextlecture{11. 10. 2019}[Irreducible topological spaces, irreducible components, generic points, minimal prime ideals.]

\lettrine{W}{e test} some things and in particular the predefined macros.
Lorem ipsum dolor sit amet, consectetur adipiscing elit, sed do eiusmod tempor incididunt ut labore et dolore magna aliqua. Dolor sed viverra ipsum nunc aliquet bibendum enim. In massa tempor nec feugiat. Nunc aliquet bibendum enim facilisis gravida. Nisl nunc mi ipsum faucibus vitae aliquet nec ullamcorper. Amet luctus venenatis lectus magna fringilla. Volutpat maecenas volutpat blandit aliquam etiam erat velit scelerisque in. Egestas egestas fringilla phasellus faucibus scelerisque eleifend. Sagittis orci a scelerisque purus semper eget duis. Nulla pharetra diam sit amet nisl suscipit. Sed adipiscing diam donec adipiscing tristique risus nec feugiat in. Fusce ut placerat orci nulla. Pharetra vel turpis nunc eget lorem dolor. Tristique senectus et netus et malesuada.





\section{Text}

Text in \enquote{quotation marks}.
A reference to \cite{lee}.

\begin{recall}
  Things you don’t know.
\end{recall}

\begin{definition}
  We call something \defemph{normal} if it is special.
\end{definition}

\begin{lemma}
  \label{generic lemma}
  Some useful stuff.
\end{lemma}

\begin{proof}
  Trivial.
\end{proof}

\begin{proposition}
  \label{generic proposition}
  More important stuff.
\end{proposition}

\begin{proof}
  Follows directly from \cref{generic lemma}.
\end{proof}

\begin{theorem}
  \label{generic theorem}
  Very important stuff.
\end{theorem}

\begin{proof}
  Similar to the proof of \cref{generic proposition}.
\end{proof}

\begin{warning}
  The technical details of the above \lcnamecref{generic theorem} are not like one expects.
\end{warning}

\begin{corollary}
  An alleged consequence.
  \qed
\end{corollary}

\begin{example}
  The most trivial example is the trivial group.
\end{example}

\begin{remark}
  An innocent observation that will ruin the exam for you.
\end{remark}





\section{Mathematics}

The Fourier transform\index{Fourier transform}\glsadd{fourier transform}:
\[
  \mathcal{F}[f](p)
  =
  \frac{1}{\sqrt{2\pi}}
  \int_{-\infty}^\infty
  f(x) e^{-ipx}
  \,\mathrm{d}x
\]
A commutative diagram\index{commutative diagram}\index{diagram!commutative}:
\[
  \begin{tikzcd}
    \Integer
    \arrow[hookrightarrow]{r}
    \arrow[hookrightarrow]{d}
    &
    \Rational
    \arrow[hookrightarrow]{d}
    \\
    \Integer[i]
    \arrow[hookrightarrow]{r}
    &
    \Rational[i]
  \end{tikzcd}
\]
\nextlecture{15. 10. 2019}[Integral ring extensions, Noether~normalization, Hilbert’s~Nullstellensätze.]
Yet another diagram:
\[
  \begin{tikzcd}
    \mathcal{C}
    \arrow[bend left = 50]{r}[above]{F}
    \arrow[bend right = 50]{r}[below]{G}
    \arrow[bend left = 50]{r}[below, name = U]{}
    \arrow[bend right = 50]{r}[above, name = D]{}
    &
    \mathcal{D}
    \arrow[MyRightarrow, from = U, to = D]{l}
  \end{tikzcd}
\]
Absolute value in normal, automatic and manual scaling:
\[
  \abs{f}
  \quad
  \abs*{\frac{\psi}{\varphi}}
  \quad
  \abs[\Bigg]{\frac{\psi}{\varphi}}
\]
Norm in normal, automatic and manual scaling:
\[
  \norm{f}
  \quad
  \norm*{\frac{\psi}{\varphi}}
  \quad
  \norm[\Bigg]{\frac{\psi}{\varphi}}
\]
Restriction with normal, automatic and manual scaling:
\[
  \restrict{(\varphi \circ \psi^{-1})}{\psi(U \cap V)}
  =
  \restrict*{(\varphi \circ \psi^{-1})}{\psi(U \cap V)}
  =
  \restrict[\big]{(\varphi \circ \psi^{-1})}{\psi(U \cap V)}
\]
Sets:
\[
  A
  \defined
  \{ x \in X \suchthat x^2 = y \}
\]
Delimiters:
\[
  \inner{v}{w}
  \quad
  \class{x}
  \quad
\]
Upright partial:
\[
  \frac{\partial f(x)}{\partial x}
\]





\section{Lists}

\noindent
Standard \texttt{enumerate}:
\begin{enumerate}
  \item
    First entry.
    \begin{enumerate}
      \item
        First subentry.
      \item
        Second subentry.
    \end{enumerate}
  \item
    Second entry.
    \begin{enumerate}
      \item
        First subentry.
      \item
        Second subentry.
    \end{enumerate}
\end{enumerate}
Standard \texttt{itemize}:
\begin{itemize}
  \item
    First entry.
    \begin{itemize}
      \item
        First subentry.
      \item
        Second subentry.
      \end{itemize}
  \item
    Second entry.
    \begin{itemize}
      \item
        First subentry.
      \item
        Second subentry.
    \end{itemize}
\end{itemize}
Standard \texttt{description}:
\begin{description}
  \item[First entry]
    Lorem ipsum.
    \begin{description}
      \item[First subentry]
        Lorem ipsum.
      \item[Second subentry]
        Lorem ipsum.
    \end{description}
  \item[Second entry]
    Lorem ipsum.
    \begin{description}
      \item[First subentry]
        Lorem ipsum.
      \item[Second subentry]
        Lorem ipsum.
    \end{description}
\end{description}

\noindent
Indented \texttt{enumerate}:
\begin{enumerate*}
  \item
    First entry.
    \begin{enumerate*}
      \item
        First subentry.
      \item
        Second subentry.
    \end{enumerate*}
  \item
    Second entry.
    \begin{enumerate*}
      \item
        First subentry.
      \item
        Second subentry.
    \end{enumerate*}
\end{enumerate*}
Indented \texttt{itemize}:
\begin{itemize*}
  \item
    First entry.
    \begin{itemize*}
      \item
        First subentry.
      \item
        Second subentry.
      \end{itemize*}
  \item
    Second entry.
    \begin{itemize*}
      \item
        First subentry.
      \item
        Second subentry.
    \end{itemize*}
\end{itemize*}
Indented \texttt{description}:
\begin{description*}
  \item[First entry]
    Lorem ipsum.
    \begin{description*}
      \item[First subentry]
        Lorem ipsum.
      \item[Second subentry]
        Lorem ipsum.
    \end{description*}
  \item[Second entry]
    Lorem ipsum.
    \begin{description*}
      \item[First subentry]
        Lorem ipsum.
      \item[Second subentry]
        Lorem ipsum.
    \end{description*}
\end{description*}
A list of equivalent statements:
\begin{equivlist}
  \item
    \label{every fd vs has a basis}
    Every finite-dimensional vector space admits a basis.
  \item
    \label{any two of the equalities}
    Any two of the following conditions hold:
    \begin{equivlist}
      \item
        $1 = 1$.
      \item
        $2 = 2$.
    \end{equivlist} 
  \item
    \label{composition is analytic}
    The composition~${\cos} \circ {\sin} \colon \Real \to \Real$ is analytic.
\end{equivlist}
A list of implications or equivalences:
\begin{implist}
  \iffitem{every fd vs has a basis}{any two of the equalities}
    Both statements are true.
  \iffitem{any two of the equalities}{composition is analytic}
    Both assertions hold true.
\end{implist}
An \texttt{enumerate} item with multiple paragraphs:
\begin{enumerate}
  \item
    \lipsum[1-2]
\end{enumerate}
